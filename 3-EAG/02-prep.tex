
\chapter{Prepping the Data and Writing the HDF5 Files}

The level-0 fits files were prepped using the IDL routine \verb+eis_prep+ available via SolarSoft
\citep{Freeland:1998} with the following options:
\begin{verbatim}
  default = 1
  save = 1
  quiet = 1
  retain = 1
  photons = 1
  refill = 0
\end{verbatim}
There are 400,000+ EIS level-0 files at present, but on a multi-core machine using the IDL bridge
all of the files can be prepped in under 24 hours. We have prepped all of the available EIS files
and saved them to standard fits files in the usual way. Some important points:
\begin{enumerate}

\item[\bf units:] As mentioned previously, the units for the output in these level-1 files is
  ``photon events'' or ``counts.''  This means that the statistical uncertainty can be estimated as
  $\sqrt{N}$. Technically the read noise should be added in, but it becomes significant only at
  very low flux levels (1--2 counts).

\item[\bf retain:] Note that the retain keyword preserves negative values. One of the jobs of
  \verb+eis_prep+ is to remove the pedestal from the CCD readout and any time-dependent dark
  current. Since the spectral windows are generally narrow, the estimate of the background can be
  too high and the subtracted intensities of the continuum can be negative. This will be dealt with
  during the fitting.

\item[\bf refill:] The warm pixel problem complicates the fitting of EIS line profiles. As
  discussed in software note \#13 in SSW, interpolating the values of missing pixels appears to
  best reproduce the original data. This option is left off during \verb+eis_prep+ so that the
  level-1 fits file preserves the information on the missing pixels. As discussed below, the
  interpolation (via the refill option) is done during the read and this data is ultimately written
  to the HDF5 file. A mask indicating which pixels have been interpolated needs to be added to the
  HDF5 file.

\end{enumerate}
Here is a code snippet related to reading the data by looping over the spectral windows.
\begin{lstlisting}[language=idl]
for iwin=0, nwin-1 do begin
  d = eis_getwindata(eis_level1_filename, iwin, /refill, /quiet)
  eis_level1_data[iwin] = ptr_new(d)
endfor
\end{lstlisting}
