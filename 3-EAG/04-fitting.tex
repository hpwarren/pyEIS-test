
\chapter{Fitting the Spectra}

Fitting of the spectra involves selecting a spectral line of interest (e.g. \ion{Fe}{12} 195.12\,\AA) from the spectral windows of the data and determining a guess on the fit parameters. The next ingredient for a fit is the selection of an optimization method\sidenote{Here we use a Python implementation of the well-known IDL method mpfit which solves the non-linear least squares problem using the Levenberg-Marquardt algorithm. The Python implementation mpfit.py is found on GitHub (https://github.com/segasai/astrolibpy/) and included in our analysis software.}.

For this we've created a set of fit templates for different spectral lines. An \verb+h5dump+ on the file shows that it contains a \verb+/template+ group for the initial guess on the fit parameters and a \verb+/parinfo+ group containing constraints on the parameters for \verb+mpfit.py+.

\begin{lstlisting}
h5dump -n fe_12_195_119.2c.template.h5
HDF5 "fe_12_195_119.2c.template.h5" {
FILE_CONTENTS {
 group      /
 group      /parinfo
 dataset    /parinfo/fixed
 dataset    /parinfo/limited
 dataset    /parinfo/limits
 dataset    /parinfo/tied
 dataset    /parinfo/value
 group      /template
 dataset    /template/component
 dataset    /template/data_e
 dataset    /template/data_x
 dataset    /template/data_y
 dataset    /template/fit
 dataset    /template/fit_back
 dataset    /template/fit_gauss
 dataset    /template/line_ids
 dataset    /template/n_gauss
 dataset    /template/n_poly
 dataset    /template/order
 dataset    /template/wmax
 dataset    /template/wmin
 }
\end{lstlisting}

 The object \verb+eis_read_template.py+ can be used to read a template file and examine the contents.

\begin{lstlisting}
from eis_read_template import eis_read_template
filename = 'fe_12_195_119.2c.template.h5'
template = eis_read_template(filename)
\end{lstlisting}

This produces the output below, showing the \verb+/parinfo+ group that contains  parameters (peak, centroid, width, background) for a double Gaussian fit along with the parameter constraints. Note that this is specific to using the \verb+mpfit+ method (see the GitHub page for more info).
\begin{lstlisting}
+ template file = fe_12_195_119.2c.template.h5
*PARAMETER CONSTRAINTS*
*              Value      Fixed            Limited                 Limits               Tied
 p[0]     57514.6647          0          1          0       0.0000       0.0000
 p[1]       195.1179          0          1          1     195.0778     195.1581
 p[2]         0.0289          0          1          1       0.0191       0.0510
 p[3]      8013.4013          0          1          0       0.0000       0.0000
 p[4]       195.1779          0          1          1     195.1378     195.2181          p[1]+0.06
 p[5]         0.0289          0          1          1       0.0191       0.0510          p[2]
 p[6]       664.3349          0          0          0       0.0000       0.0000
 \end{lstlisting}

 Next you'll want to prep the data for fitting. Once you've read in a template file, you can use the central wavelength to find the desired spectral window in the data using \verb+eis_read_raster+ as shown in the previous chapter.

\begin{lstlisting}
from eis_read_raster import eis_read_raster
from eis_read_template import eis_read_template

# input data and template files
file_data     = 'eis_20190404_131513.data.h5'
file_template = 'fe_12_195_119.2c.template.h5'

# read fit template
template = eis_read_template(file_template)

# get central wavelength
wmin = template.template['wmin']
wmax = template.template['wmax']
wave = wmin + (wmax-wmin)*0.5

# read raster
raster = eis_read_raster(file_data, wave)
ints   = raster.data['data']
wave   = raster.data['wave']
corr   = raster.data['wave_corr']
\end{lstlisting}

Prepping of the data can be handled at various levels of sophistication at the user's discretion, however, at a minimum it should include handling bad values\sidenote{Negative values are a result of the background subtraction.} in the raster, correcting for the wavelength offsets\sidenote{This is from thermal shifts in the CCD and normal degradation of being in space over time?...}, and computing the errors on the intensities\sidenote{The square root of the counts is a good first-order approximation.}.

\begin{lstlisting}
# get dimensions
ndata = ints.shape
nx    = ndata[0]
ny    = ndata[1]
nz    = ndata[2]

# bad data correction
bad = np.where(ints<0)
ints[bad] = 0.0

# compute error on counts
errs = np.sqrt(ints)

# wavelength correction
newwave = np.zeros(ndata)
for i in range(nx):
    for j in range(ny):
        newwave[i,j,::] = wave-corr[i,j]
wave = newwave
\end{lstlisting}

Now on to the fitting! Now that you have a fit template and the data elements, you can perform a fit of the entire raster by calling \verb+eis_fit_raster.py+\sidenote{Here's what's happening under the hood. The object eis-fit-raster calls eis-scale-guess to scale the initial parameter guess to the data, then calls eis-mpfit to implement the Levenberg-Marquardt fitting. The module eis-fit-deviates contains the callable function that returns the fit deviates computed from a model function for eis-mpfit.}. The fit results can be saved and read back using \verb+eis_save_fit.py+ and \verb+eis_read_fit.py+.

\begin{lstlisting}
from eis_fit_raster import eis_fit_raster
from eis_save_fit import save_fit
from eis_read_fit import read_fit

# fit profile
parinfo  = template.parinfo
template = template.template
fit      = eis_fit_raster(wave, ints, errs, template, parinfo)

# save fit output
fit = fit.fit
file_fit = save_fit(fit, file_data)

# read fit output back from file
fit = read_fit(file_fit[0])
\end{lstlisting}

The output fit parameters are stored to a dictionary.

\begin{lstlisting}
background   float64      (512, 87, 1)
centroid     float64      (512, 87, 2)
chi2         float64      (512, 87)
component    int64        1
e_background float64      (512, 87, 1)
e_centroid   float64      (512, 87, 2)
e_int        float64      (512, 87, 2)
e_peak       float64      (512, 87, 2)
e_width      float64      (512, 87, 2)
int          float64      (512, 87, 2)
line_ids     object       (2,)
n_gauss      int16        1
n_poly       int16        1
params       float64      (512, 87, 7)
peak         float64      (512, 87, 2)
perror       float64      (512, 87, 7)
status       float64      (512, 87)
wavelength   float64      (512, 87, 24)
width        float64      (512, 87, 2)
\end{lstlisting}

The above steps are illustrated in the example routine \verb+eis_fit_example.py+, which produces a plot like the one shown below.

\begin{figure}[t]
  \centerline{\includegraphics[scale=0.25]{figures/fit_example.pdf}}
  \caption{Fit Examples.}
  \label{fig:fit_example}
\end{figure}
